\let\negmedspace\undefined{}
\let\negthickspace\undefined{}

\documentclass{beamer}
\usepackage{amsthm}
 \usepackage{gensymb}
 \usepackage{polynom}
\usepackage{amssymb}
%
  \usepackage{stfloats}
\usepackage{bm} 
 \usepackage{longtable}
 \usepackage{enumitem}
 \usepackage{mathtools}
 \usepackage{tikz}
  %  \usepackage[breaklinks=true]{hyperref}
  \usepackage{listings}
\usepackage{color}                                            
\usepackage{array}                                            
\usepackage{longtable}                                        
\usepackage{calc}                                             
     \usepackage{multirow}                                         
     \usepackage{hhline}                                           
     \usepackage{ifthen}                                           
     \usepackage{lscape}     
\usetheme{CambridgeUS}
\DeclareMathOperator*{\Res}{Res}
\DeclareMathOperator*{\equals}{=}
\renewcommand\thesection{\arabic{section}}
\renewcommand\thesubsection{\thesection.\arabic{subsection}}
\renewcommand\thesubsubsection{\thesubsection.\arabic{subsubsection}}
% \renewcommand\thesectiondis{\arabic{section}}
% \renewcommand\thesubsectiondis{\thesectiondis.\arabic{subsection}}
% \renewcommand\thesubsubsectiondis{\thesubsectiondis.\arabic{subsubsection}}
\hyphenation{op-tical net-works semi-conduc-tor}
 \def\inputGnumericTable{}                                 %%
\lstset{ 
frame=single,
breaklines=true,
columns=fullflexible
}

% \newtheorem{theorem}{Theorem}[section]
% \newtheorem{problem}{Problem}
% \newtheorem{proposition}{Proposition}[section]
% \newtheorem{lemma}{Lemma}[section]
% \newtheorem{corollary}[theorem]{Corollary}
% \newtheorem{example}{Example}[section]
% \newtheorem{definition}[problem]{Definition}
\newcommand{\BEQA}{\begin{eqnarray}}
\newcommand{\EEQA}{\end{eqnarray}}
\newcommand{\define}{\stackrel{\triangle}{=}}
\newcommand*\circled[1]{\tikz[baseline= (char.base)]{
    \node[shape=circle,draw,inner sep=2pt] (char) {#1};}}
\bibliographystyle{IEEEtran}
\providecommand{\mbf}{\mathbf}
\providecommand{\pr}[1]{\ensuremath{\Pr\left(#1\right)}}
\providecommand{\qfunc}[1]{\ensuremath{Q\left(#1\right)}}
\providecommand{\sbrak}[1]{\ensuremath{{}\left[#1\right]}}
\providecommand{\lsbrak}[1]{\ensuremath{{}\left[#1\right.]}}
\providecommand{\rsbrak}[1]{\ensuremath{{}\left[#1\right.]}}
\providecommand{\brak}[1]{\ensuremath{\left(#1\right)}}
\providecommand{\lbrak}[1]{\ensuremath{\left(#1\right.)}
\providecommand{\rbrak}[1]{\ensuremath{\left[#1\right.]}}}
\providecommand{\cbrak}[1]{\ensuremath{\left\{#1\right\}}}
\providecommand{\lcbrak}[1]{\ensuremath{\left\{#1\right.}}
\providecommand{\rcbrak}[1]{\ensuremath{\left.#1\right\}}}
\theoremstyle{remark}
\newtheorem{rem}{Remark}
\newcommand{\sgn}{\mathop{\mathrm{sgn}}}
\providecommand{\abs}[1]{\left\vert#1\right\vert}
\providecommand{\res}[1]{\Res\displaylimits_{#1}} 
\providecommand{\norm}[1]{\left\lVert#1\right\rVert}
\providecommand{\mtx}[1]{\mathbf{#1}}
\providecommand{\mean}[1]{E\left[ #1 \right]}
\providecommand{\fourier}{\overset{\mathcal{F}}{ \rightleftharpoons}}
\providecommand{\system}{\overset{\mathcal{H}}{ \longleftrightarrow}}
% \newcommand{\solution}{\noindent \textbf{Solution: }}
\newcommand{\cosec}{\,\text{cosec}\,}
\newcommand*{\permcomb}[4][0mu]{{{}^{#3}\mkern#1#2_{#4}}}
\newcommand*{\perm}[1][-3mu]{\permcomb[#1]{P}}
\newcommand*{\comb}[1][-1mu]{\permcomb[#1]{C}}
\renewcommand{\thetable}{\arabic{table}} 
\providecommand{\dec}[2]{\ensuremath{\overset{#1}{\underset{#2}{\gtrless}}}}
\newcommand{\myvec}[1]{\ensuremath{\begin{pmatrix}#1\end{pmatrix}}}
\newcommand{\mydet}[1]{\ensuremath{\begin{vmatrix}#1\end{vmatrix}}}
\numberwithin{equation}{section}
\numberwithin{figure}{section}
\numberwithin{table}{section}
\makeatletter
\@addtoreset{figure}{problem}
\makeatother
\let\StandardTheFigure\thefigure{}
\let\vec\mathbf{}
\def\putbox#1#2#3{\makebox[0in][l]{\makebox[#1][l]{}\raisebox{\baselineskip}[0in][0in]{\raisebox{#2}[0in][0in]{#3}}}}
     \def\rightbox#1{\makebox[0in][r]{#1}}
     \def\centbox#1{\makebox[0in]{#1}}
     \def\topbox#1{\raisebox{-\baselineskip}[0in][0in]{#1}}
     \def\midbox#1{\raisebox{-0.5\baselineskip}[0in][0in]{#1}}
\vspace{3cm}
\title{Assignment 13 Papoulis ex 11.8}
\author{Gunjit Mittal (AI21BTECH11011)}
\date{\today}
\logo{\large \LaTeX}
\begin{document} 
\begin{frame}
  \titlepage{}
\end{frame} 
\logo{}
\begin{frame}{Outline} 
  \tableofcontents
\end{frame}
% Download all python codes from 
% \begin{lstlisting}
% https://github.com/GunjitMittal/Assignment6/tree/main/Assignment6/code
% \end{lstlisting}     
% Download all latex codes from 
% \begin{lstlisting}
% https://github.com/GunjitMittal/Assignment6/tree/main/Assignment6 
% \end{lstlisting} 
\section{Question} 
\begin{frame}{Question}
    Show that if X (t) is WSS and 
    \begin{align*}
        X_T(\omega) = \int^{\frac{T}{2}}_{\frac{-T}{2}} x(t)e^{-j\omega t} dt
    \end{align*}
         Then
    \begin{align*}
         E\cbrak{\frac{\partial}{\partial T}{|X_T(\omega)|}^2} = \int_{-T}^T R_{\tau}(\tau)e^{-j\omega\tau} d\tau
    \end{align*}
\end{frame}
\section{Solution} 
\begin{frame}{Solution} 
  \begin{align}
  &X_T(\omega) = \int^{\frac{T}{2}}_{\frac{-T}{2}} x(t_1)e^{-j\omega t_1} dt_1\\
  &X_T^*(\omega) = \int^{\frac{T}{2}}_{\frac{-T}{2}} x^*(t_2)e^{-j\omega t_2} dt_2\\
  &|X_T(\omega)|^2 = \int^{\frac{T}{2}}_{\frac{-T}{2}} \int^{\frac{T}{2}}_{\frac{-T}{2}} x(t_1)x^*(t_2)e^{-j\omega (t_1-t_2)} dt_1 dt_2\\
  &E(|X_T(\omega)|^2) = \int^{\frac{T}{2}}_{\frac{-T}{2}} \int^{\frac{T}{2}}_{\frac{-T}{2}} E(x(t_1)x^*(t_2))e^{-j\omega (t_1-t_2)} dt_1 dt_2\\
  &~~~~~~~~~~~~~~~~= \int^{\frac{T}{2}}_{\frac{-T}{2}} \int^{\frac{T}{2}}_{\frac{-T}{2}} R(t_1,t_2)e^{-j\omega (t_1-t_2)} dt_1 dt_2
\end{align}
\end{frame}
\begin{frame}
Let $ \tau = t_1 - t_2$
\begin{align}
  E(|X_T(\omega)|^2) =  \int^{T}_{-T} (T-|\tau|)R(\tau)e^{-j\omega \tau} d\tau
\end{align}
Differentiating w.r.t. T 
\begin{multline}
  \frac{\partial}{\partial T}E(|X_T(\omega)|^2) =  (T-|T|)R(T)e^{-j\omega T} - (T-|-T|)R(-T)e^{j\omega T} \\+ \int^{T}_{-T} R(\tau)e^{-j\omega \tau} d\tau
\end{multline}
\begin{align}
  &\frac{\partial}{\partial T}E(|X_T(\omega)|^2) = 0 - 0 + \int^{T}_{-T} R(\tau)e^{-j\omega \tau} d\tau = \int^{T}_{-T} R(\tau)e^{-j\omega \tau} d\tau\\
  &\implies E\brak{\frac{\partial}{\partial T}|X_T(\omega)|^2} = \int^{T}_{-T} R(\tau)e^{-j\omega \tau} d\tau
\end{align}
\end{frame}
\end{document}     